\documentclass{resume}
\usepackage[english]{babel}
\usepackage[left=0.4in,top=0.4in,right=0.4in,bottom=0.4in]{geometry} % Document margins
\newcommand{\tab}[1]{\hspace{.2667\textwidth}\rlap{#1}} 
\newcommand{\itab}[1]{\hspace{0em}\rlap{#1}}
\name{Даниил Спиридонов}
\address{+7(936) 505-1245 \\ Москва, Россия} 
\address{\href{mailto:sprv-4@yandex.ru}{sprv-4@yandex.ru} \\ \href{https://github.com/notaskynet}{github.com/notaskynet}}  %
\begin{document}

%----------------------------------------------------------------------------------------
%	ABOUT ME
%----------------------------------------------------------------------------------------
\begin{rSection}{ОБО МНЕ}
Студент 3 курса ОП "Компьютерная безопасность" в НИУ ВШЭ. Развиваюсь в ML/DS направлении, участвую в хакатонах.
\end{rSection}

%----------------------------------------------------------------------------------------
%	EDUCATION SECTION
%----------------------------------------------------------------------------------------
\begin{rSection}{ОБРАЗОВАНИЕ}
{\bf Специалитет, Образовательная программа "Компьютерная безопасность" \\ Национальный исследовательский университет Высшая школа экономики (НИУ ВШЭ)}, \hfill {Ожидается в 2028}\\
\textit{Московский институт электроники и математики им. А.Н. Тихонова} 
\end{rSection}

%----------------------------------------------------------------------------------------
% TECHNICAL SKILLS	
%----------------------------------------------------------------------------------------
\begin{rSection}{НАВЫКИ}
\begin{tabular}{ @{} >{\bfseries}l @{\hspace{6ex}} l }
Программирование: & Python, C++ \\
Машинное обучение: & CatBoost, XGBoost, Transformers, LangChain, PyTorch, Scikit-Learn \\
Аналитика данных: & Pandas, NumPy, SQL \\
Инструменты: & Docker, Git, Airflow, MLflow, GitHub CI/CD \\
\end{tabular}
\end{rSection}

%----------------------------------------------------------------------------------------
% EXPERIENCE
%----------------------------------------------------------------------------------------
\begin{rSection}{ОПЫТ РАБОТЫ}

\textbf{ML Engineer}  
\textit{РТ-Информационная безопасность} | Дек 2024 – по настоящее время  
\begin{itemize}
    \itemsep -3pt {} 
    \item Разработал и развернул модели машинного обучения для работы инцидентами информационной безопасности.
    \item Проектировал и реализовывал пайплайны сбора и обработки данных.
\end{itemize}

\end{rSection}

%----------------------------------------------------------------------------------------
% PROJECTS
%----------------------------------------------------------------------------------------
\begin{rSection}{ПРОЕКТЫ}

\textbf{Claw Engine: Модуль Искусственного Интеллекта}  
\textit{НИУ ВШЭ}  
\begin{itemize}
    \itemsep -3pt {} 
    \item Адаптировал модель на основе графовых нейронных сетей для решения конкретной задачи.
    \item Реализовал сбор данных и визуализацию для оценки эффективности модели.
    \item Разработал суммаризатор отзывов в рамках модуля ИИ, анализирующий отзывы и выдающий структурированное содержание по пунктам, с использованием LangChain для обработки естественного языка.
\end{itemize}

\textbf{Анализатор вакансий на hh.ru}  
\textit{Pet-Проект}  
\begin{itemize}
    \itemsep -3pt {} 
    \item Разработал парсер для сбора вакансий с hh.ru, обходя анти-бот меры.
    \item Разработал суммаризацию набора квалификаций из множества вакансий.
    \item Разработал визуализацию для анализа трендов в требуемых навыках и условиях работы.
\end{itemize}

\end{rSection}

%----------------------------------------------------------------------------------------
%	HACKATHONS
%----------------------------------------------------------------------------------------
\begin{rSection}{ХАКАТОНЫ}
\begin{itemize}
    \itemsep -3pt {} 
    \item Участвовал в **Ozon Tech Hackathon**, разработал модель на основе **VGG16** для детекции сигарет на изображениях.
    \item Участвовал в **HSE AI Assistant Hack**, адаптировал **GigaChat от Сбера** для конкретной задачи.
\end{itemize}
\end{rSection}

\end{document}